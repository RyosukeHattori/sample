\section{正則化と次元圧縮がスコア予測に与える影響の評価実験}
正則化法と次元圧縮法のスコア予測精度を検証し,適切なWord2Vecのハイパーパラメータを明らかにする実験を行う.
また,正則化法と次元圧縮法の学習時間を計測した後, $2$つの手法の偏回帰係数の相関を検証する.

本研究では,Book-Crossingデータセットを用いる.
このデータセットはユーザが書籍に付与した10段階のスコアを集計したものであり,ユーザ数278,858,書籍数271,379,総スコア数383,852である.
また,重回帰分析にはRのglmnetパッケージを使用する.
適切なWord2Vecのハイパーパラメータの検証では, $10$種類の圧縮次元数と$5$種類のウィンドウサイズの組み合わせ計$50$種類について実験を行う.
なお,他のハイパーパラメータは,最小出現数0,Skip-gram学習モデルとし,その他は使用プログラム\cite{w2v}のデフォルトを用いる.
評価しているユーザ数が多い書籍上位$100$件を目的変数,残りの全書籍を説明変数に選択し,目的変数毎に回帰式を立てる.
なお,説明変数の欠損値はスコアの中央値である$5.5$で補間する.

予測精度には平均絶対誤差(MAE: Mean Absolute Error)$MAE = \frac{1}{N} \sum_{i=1}^N |\hat{y_i} - y_i| \label{mae} $を用いる.
ここで,$N$はテストデータ数,$\hat{y_i}$は予測値,$y_i$は真値を表す.
一般に,モデルの汎化誤差の検証には交差検証法が用いられるが,目的変数を評価しているユーザが学習データとテストデータの両方に適切な数含まれることを保証できないため,本研究ではブートストラップ法を適用する.