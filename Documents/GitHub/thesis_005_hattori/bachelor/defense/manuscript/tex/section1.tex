\section{はじめに}
協調フィルタリングは推薦システムの手法のひとつで,購入履歴やレーティングなどから類似ユーザの発見やスコア予測を行う.
また,推薦理由を説明する情報をユーザに提供することを,推薦の透明性という.
協調フィルタリングにおける推薦の透明性に関する研究\cite{sinha}\cite{ged}はメモリベース法が主流であるが,推薦のたびに近傍探索を行う必要があるため計算コストが高くなる傾向にある.

本研究では,低計算コストでの推薦の透明性の実現を目指し,モデルベース協調フィルタリングの手法のひとつである重回帰分析の適用を試みる.
目的変数を評価しているユーザのデータのみの使用を想定するため,学習データ数が少なくなり過学習になる可能性がある.
そこで,解決法である正則化と次元圧縮について,スコア予測精度を検証する実験を行う.
正則化法にはL1・L2正則化を適用し,次元圧縮法には自然言語処理技術のひとつであるWord2Vec\cite{mik}を用いた手法を提案する.

