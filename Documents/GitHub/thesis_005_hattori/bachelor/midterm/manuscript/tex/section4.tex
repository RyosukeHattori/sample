\section{おわりに}
本研究では,word2vecで次元圧縮したユーザ-アイテム行列を重回帰分析し,得られたパラメータから元の次元のパラメータを推定する手法を提案している.
Book crossingデータセットを用いた実験結果から,直接法と提案法の予測精度と推定したパラメータには違いがないことを確認している.
また,推定したパラメータの比較より,提案法は直接法にL2正則化(ridge)を適用した際と同様の効果を示していると考える.

今後は,説明変数の数を増やし,提案法の予測精度や直接法に正則化を適用した場合との計算コストの違いを明らかにすることを検討している.
また,適切なword2vecのハイパーパラメータについても検証していきたい.