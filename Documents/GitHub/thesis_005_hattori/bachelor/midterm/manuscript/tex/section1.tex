\section{はじめに}
既存の賃貸物件の価格は、築年数やロケーション、などの変えられない属性の影響を強く受けるために、部屋ごとの価格差を出しにくい課題がある。
本研究では、不動産業者が物件の部屋ごとの価格差をつけやすくしたり、より大きな付加価値を提供できるための可視化モデルを提案する。
提案手法として、本研究ではシャムネットワークを用いて、家賃や築年数、専有面積などを入力し、間取り画像の特徴ベクトルの類似に応じたクラスタリングを行う。
シャムネットワーク  \cite{jj}  とは,出力が結合された2つの同一のサブネットワークで構成されたもので、統一されたフレームワーク内でテクスチャ特徴及びメトリックを共同で学習することができる技術である。
本研究では、学習したモデルに新規のデータを入力し、そのデータの分布場所及び、周辺の物件を可視化する。
また、分布の位置が、変えられる変数(家賃や、共益費まど)を変えた時の分布遷移を可視化する。


用いるデータセットについて本研究では、国立情報学研究所が株式会社LIFULから提供を受けて研究者に提供しているデータセットを使用する。
データセットは2015年9月時点から2017年4月1日までの賃料,面積,立地(市区町村,郵便番号,最寄り駅,徒歩分),築年数,間取り,建物構造,諸設備などの全国約533万件の賃貸物件データ及び、それら物件に紐付けされた、大横120ピクセル×縦120ピクセルのJPEG形式の間取り図や室内写真などの画像データ約8,300万ファイルを用いる。
本研究では、路線、徒歩距離、建物面積/専有面積、建物階数(地上)、築年月、部屋階数、間取部屋数、間取部屋種類及び、賃料/価格、賃料+管理費が欠損している物件を取り除いたデータ及び、間取り画像のみを使用した。
また、河合伸治 \cite{kwa} の論文を参考にデータを加工した。